\documentclass[10pt]{article}
\usepackage{palatino}
\usepackage[utf8]{inputenc}   % pro unicode UTF-8
\usepackage[IL2]{fontenc}
\usepackage[czech]{babel}
\usepackage{a4wide}
\usepackage{amsmath,amsfonts,amssymb,graphicx}
\usepackage{wrapfig}
\usepackage{hyperref}
\usepackage{fontawesome}


\hypersetup{
    colorlinks=true,
    linkcolor=blue,
    filecolor=blue,
    urlcolor=blue,
    citecolor=blue
}


\begin{document}

\begin{center}
\textbf{\LARGE Ing. Josef Štěpáník} \\
\end{center}

\textbf{Osobní informace} 

\begin{tabbing}
    \hspace{6cm}\= \kill
    Datum narození: \> 5. 4. 1988 \\
    Adresa trvalého bydliště: \> Kejnice 6, 341 01 Horažďovice \\
    Místo pobytu: \> Žitná 1731/44, 621 00 Brno-Řečkovice \\
    Telefon: \> \faPhone \ +420 724 160 236 \\
    Email: \> \faEnvelope \ \href{mailto:stepjo.hd@gmail.com}{stepjo.hd@gmail.com} \\
    LinkedIn: \>  \faLinkedin \ \href{https://www.linkedin.com/in/josef-štěpáník-30106174/}{Josef Štěpáník} \\
    GitHub: \>  \faGithub \ \href{https://github.com/JosefStepanik}{Josef Stepanik} \\
    Instagram: \>  \faInstagram \ \href{https://www.instagram.com/josef_stepanik/}{josef\_stepanik}
\end{tabbing}

\noindent\hrulefill
\vskip 0,4cm

\textbf{Profesní shrnutí}
\vskip 0,3cm

\indent Popsal bych se jako praktický technik a vývojář se zkušenostmi v návrhu a realizaci jednoúčelových zařízení a~měřicí techniky. 
Schopen samostatně i v týmu řešit technické problémy, od~návrhu hardwaru po základní programování a automatizaci. \\
\indent Mezi mé profesní vlastnosti patří komunikativnost, ochota se učit novým věcem, nekonfliktnost. Nechybí mi sportovní, motivační a soutěživý duch.
\vskip 0,3cm

\noindent\hrulefill
\vskip 0,4cm

\textbf{Dosavadní zaměstnání}

\begin{itemize}
\item \textbf{IQS nano s.r.o.} [2022 - doposud], pozice: technik, programátor. Vývoj a výroba jednoúčelových zařízení pro výzkum a výrobu v oblasti nanotechnologiích.
\item \textbf{Activair s.r.o.} [2017 - 2022], pozice: elektroinženýr. Návrh a vývoj zařízení v~prostředí vakuové techniky, včetně manipulátorů a distribučních systémů.
\item \textbf{CEITEC-VUT} [2019 - 2021], pozice: technik, manažer laboratoře, výzkumná skupina: Charakterizace materiálů a pokročilé povlaky - Jozef Kaiser. 
Vývoj jednoúčelových zařízení pro~testování vzorků ve výzkumu. 
\item \textbf{Rohde \& Schwarz} závod Vimperk s.r.o. [2014 - 2017], pozice: produktový inženýr. Podpora vývoje a výroby spektrálních analyzátorů, zajištění měřicích a výrobních postupů. 
\end{itemize}

\noindent\hrulefill
\vskip 0,4cm


\textbf{Vzdělání}

\begin{itemize}
    \item [\faGraduationCap]\textbf{ČVUT v Praze, Fakulta elektrotechnická }
    \begin{itemize}
        \item studijní program Komunikace, multimedia a elektronika, studijní obor Multimediální technika - titul Ing. [2011 - 2014]
        \item Magisterská práce na téma Doostření obrazu pomocí Curvelet transformace.
        \item studijní program Elektronika, studijní obor Elektronika a sdělovací technika - titul Bc. [2007 - 2011]
    \end{itemize}
    \item[\faGraduationCap] \textbf{Gymnázium Jaroslava Vrchlického, Klatovy }
    \begin{itemize}
        \item Všeobecné gymnázium [2004 - 2007]
    \end{itemize}
\end{itemize}

\noindent\hrulefill
\vskip 0,4cm

\clearpage

\textbf{Dovednosti}

\begin{itemize}
    \item \textbf{PC odborné:} 
    \begin{itemize}
        \item práce v simulačních programech Matlab, Microwave Office (RF/Microwave curciut desing), CST Microwave Studio (3D EM Simulation).
        \item tvorba aplikací pomocí Visual Studio Code v jazycích Python, PyQt, Wiring (C++), Angular, HTML5. Využívání verzovacího systému GIT.
        \item tvorba desek plošných spojů v Autodesk Eagle.
        \item kooperace s konstruktéry v programech Autodesk Fusion360.
        \item tvorba elektronických a procesních schémat v SW Proficad.
    \end{itemize}
    \item \textbf{Měřící technika:} obsluha a integrace běžných laboratorních měřících přístrojů (zdroj, osciloskop, spektrální analyzátor atd).
    \item \textbf{Automatizace:} návrh prvků automatizace pro jednoúčelové stroje, tvorba obslužných programů dle standardu IEC 61131-3, 
    komunikace s protokoly Modbus, EtherCat, RS485/RS232, OPC UA, atd. Vytváření uživatelských rozhraní.
    \item \textbf{3D tisk:} zkušenosti s tiskárnami Funmat HT (filamenty Peek, Pekk a další materiály), Markforged MarkTwo (nylon + vlákna karbon, kevlar, sklolaminát), Formlabs Form 3 (LFS), DeltiQ (standardní filamenty) a příprava tiskových front v různém SW. 
    obsluha a integrace laboratorních měřících přístrojů (zdroj, osciloskop, spektrální analyzátor).
    \item \textbf{Obecný SW:} MS Office, \uv\LaTeX, Affinity Photo, Adobe Photoshop, DaVinci Resolve.
    \item \textbf{Jazyky:} angličtina – střední úroveň, němčina – nižší úroveň.
    \item \textbf{Řidičský průkaz:} skupina B.
\end{itemize}

\noindent\hrulefill
\vskip 0,4cm

\end{document}