\documentclass[12pt]{article}
\usepackage{palatino}
% \usepackage[latin2]{inputenc} % pro iso8859-2
\usepackage[utf8]{inputenc}   % pro unicode UTF-8
% \usepackage[cp1250]{inputenc} % pro win1250
\usepackage[IL2]{fontenc}
\usepackage[czech]{babel}
\usepackage{a4wide}
\usepackage{amsmath,amsfonts,amssymb,graphicx}
\usepackage{wrapfig}
\usepackage{hyperref}

\begin{document}

\begin{center}
\textbf{\LARGE Curriculum Vitae} \\
\end{center}

\begin{wrapfigure}{r}{0.35\textwidth}
\vspace{-1cm}
%\includegraphics[width=0.2\textwidth]{foto.jpg}
\end{wrapfigure}



\vskip 0,5cm

\textbf{Osobní informace} 
\begin{tabbing}
    \hspace{6cm}\= \kill
    Jméno: \> Ing. Josef Štěpáník \\
    Adresa trvalého bydliště: \> Kejnice 6, 341 01 Horažďovice \\
    Místo pobytu: \> Žitná 1731/44, 621 00 Brno-Řečkovice \\
    Telefon: \> +420 724 160 236 \\
    Email: \> \href{mailto:stepjo.hd@gmail.com}{stepjo.hd@gmail.com} \\
    Datum narození: \> 5. 4. 1988 \\
    Místo narození: \> Sušice \\
\end{tabbing}
\vskip 1cm


\textbf{Dosavadní zaměstnání}

\begin{itemize}
\item IQS nano s.r.o. [2022-doposud], pozice: technik, programátor, vývoj a výroba jednoúčelových zařízení pro výzkum a vývoj v oblasti nanotechnologií.


\item Activair s.r.o., pobočka Brno [2017-2022], pozice: elektroinženýr, vedoucí vývojového týmu, návrh a vývoj jednoúčelových zařízení z prostředí vakuové techniky (depoziční aparatury, víceosé manipulátory, distribuce plynů, chemický průmysl) 

\item 
Středoevropský technologický institut Vysokého učení technického 
(CEITEC-VUT) Brno [2019-2021], pozice: technik,manažer laboratoře,  výzkumná skupina: Charakterizace materiálů a pokročilé povlaky - Jozef Kaiser. 
Vývoj jednoúčelových zařízení pro testování vzorků ve výzkumu. 

\item Rohde \& Schwarz závod Vimperk s.r.o. [2014-2017], pozice: produktový inženýr, podpora výroby spektrálních a vektorových analyzátorů, vylepšování jejich parametrů. Součinnost při vývoji nového spektrálního analyzátoru, měřících a výrobních postupů,zajištění typových zkoušek a uvedení do sériové výroby. \\
Absolvování stáže v sídle vývoje v Mnichově [2016].
 
\end{itemize}

\textbf{Vzdělání}

\begin{itemize}
\item České vysoké učení technické v Praze, Fakulta elektrotechnická [2011 - 2014], studijní program: Komunikace, multimedia a elektronika, studijní obor: Multimediální technika - titul Ing. \\
Magisterská práce na téma Doostření obrazu pomocí Curvelet transformace.

\item České vysoké učení technické v Praze, Fakulta elektrotechnická [2007 - 2011], studijní program: Elektronika, studijní obor: Elektronika a sdělovací technika - titul Bc.


\item Gymnázium Jaroslava Vrchlického, Klatovy, 
obor: všeobecné gymnázium [2004 - 2007].
\end{itemize}

\textbf{Zkušenosti a dovednosti}

\begin{itemize}
\item Jazyky: angličtina - střední úroveň, němčina - nižší úroveň
\item Řidičský průkaz - skupina B
\item Odborné zkušenosti: 
\begin{description}
\item[$\diamond$]Měřící technika - obsluha a integrace běžných laboratorních měřících přístrojů (zdroj, osciloskop, spektrální analyzátor atd).
\item[$\diamond$]Automatizace - Návrh prvků automatizace pro jednoúčelové stroje, 
tvorba obslužných programů dle standardu IEC 61131-3, komunikace s protokoly Modbus, EtherCat, RS485/RS232, OPC UA, atd. Vytváření uživatelských rozhraní.
\item[$\diamond$]3D tisk - zkušenosti s tiskárnami Funmat HT (filamenty Peek, Pekk a další materiály), Markforged MarkTwo (nylon + vlákna karbon, kevlar, sklolaminát), Formlabs Form 3 (LFS), DeltiQ (standardní filamenty) a příprava tiskových front v různém SW. 
\item[$\diamond$]PC odborné - práce v simulačních programech Matlab, Microwave Office (RF/Microwave curciut desing), CST Microwave Studio (3D EM Simulation).\\
- tvorba aplikací pomocí Visual Studio Code v jazycích Python, Angular, Wiring (C, C++), HTML5. Využívání verzovacího systému GIT.\\
- tvorba desek plošných spojů v Autodesk Eagle.\\
- kooperace s konstruktéry v programech Autodesk Fusion360.\\
- tvorba elektronických a procesních schémat v SW Proficad.
\item[$\diamond$]Vyhláška 50 § 6, 7, 8 a 10
\item[$\diamond$]PC administrativa - MS Office365, \uv\LaTeX, SAP.

\item[$\diamond$]PC ostatní - zpracování fotek a reklamních materiálů v prostředí Adobe Photoshop, Lightroom, Affinity Photo, Capture One. Tvorba videí v DaVinci Resolve.
\end{description}       
\item Vlastnosti: práce ve výzkumném týmu, komunikativnost, ochota se učit novým věcem, nekonfliktnost, dobré vycházení v kolektivu. Sportovní, motivační a soutěživý duch.

\end{itemize}


\end{document}